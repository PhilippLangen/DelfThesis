\section*{Motivation}
Präzise und effiziente Suchwerkzeuge sind essenziell um große Datenmengen für einen Nutzer sinnvoll verwertbar zu machen. Dies gilt insbesondere auch im Bereich der Bildersuche. Die klassische Bildersuche basiert auf vom Nutzer formulierten Anfragen, mit deren Hilfe das Suchsystem eine Reihen an passenden Bildkandidaten zusammenstellt und zurückgibt. Hierbei nutzt das System eine Reihe von Zusatzinformationen, sogenannten Metadaten, wie Tags, Titel, Aufnahmeort oder Datum. Eine alternativer Ansatz der Suche, der in dieser Arbeit behandelt wird, ist die inhaltsbasierte Bildersuche (engl. Content-Based Image Retrieval kurz CBIR). Hierbei werden vom Nutzer keine Anfragen formuliert. Stattdessen dient ein Bild als Suchanfrage. Ziel ist es Bilder mit gleichem oder ähnlichem Bildinhalt als Ergebnis zurückzugeben. Ein Vorteil dieser Herangehensweise ist, dass der Nutzer keine Informationen über den Inhalt des Suchbildes benötigt. Das System arbeitet ausschließlich mit den Pixelinformationen der Bilder. Ein weiterer Vorteil ist daher, dass weder im Suchbild noch in der Suchdatenbank Metadaten zu den Bildern vorhanden sein müssen, was die Einsatzmöglichkeiten von inhaltsbasierter Suche sehr flexibel gestaltet. Im Folgenden wird die inhaltsbasierte Suche auch als Image Retrieval bezeichnet.
\\\\
Das Anwendungsgebiet dieser Arbeit ist die Suche auf historischen Bildern. Diese weit gefasste Domäne ist besonders herausfordernd, da sie sehr heterogene Daten enthält. Dabei gibt es nicht nur sehr große Unterschiede in den abgebildeten Bildinhalten, wie Gebäude, Naturaufnahmen oder Portraits, sondern auch in den verwendeten Aufnahmetechniken. Durch die Fortschritte der Aufnahmetechnologie können historische Bilder sowohl in Form von Zeichnungen oder Malerei, aber auch als Druck oder in anfänglichen Formen der Photographie vorliegen. Da Metadaten zu historischen Bildern erst bei der Digitalisierung hinzugefügt werden können sind diese oft gar nicht oder nur lückenhaft vorhanden. Dies macht die inhaltsbasierte Suche für diese Domäne im Vergleich zur klassischen Bildersuche zu einer besonders geeigneten Methode. Mit der Umsetzung einer unterstützenden Suche für das UrbanHistory4D Projekt \cite{urbanhistory4d} ergibt sich ein konkreter Anwendungsfall für diese Arbeit. Das UrbanHistory4D Projekt befasste sich mit der Erstellung interaktiver Stadtkarten, die dem Nutzer erlauben sich an verschiedenen Plätzen innerhalb der Städte historische Bilder dieser anzeigen zu lassen. Das Akkumulieren und Zuordnen von historischen Bildern zu diesen Plätzen ist ein wesentlicher Arbeitsanteil bei der Erstellung dieser Karten. Image Retrieval Systeme bieten hier die Chance den Suchaufwand für die Ersteller signifikant zu reduzieren. Dabei handelt es sich um einen aktiven Forschungsbereich, in dem momentan unterschiedliche Suchsysteme analysiert werden. 
\\\\
Das Image Retrieval Verfahren DELF (attentive DEep Local Features) \cite{delf}, welches in dieser Arbeit untersucht wird ist ein Deep Learning Ansatz. Durch den raschen Fortschritt im Bereich tiefer neuronaler Netzwerkarchitekturen der letzten Jahre erfreuen sich gelernte Ansätze immer größerer Beliebtheit. DELF erzielt auf bekannten Benchmarkdatensätzen wie Oxford5k \cite{oxford5k} und Paris6k \cite{paris6k} sehr gute Ergebnisse. Besonders gut schneidet DELF im Vergleich auf dem eigens erstellten Google Landmarks Datensatz \cite{landmarks} ab. Dieser enthält mit über 1 mio. Bilder und 13k unterschiedlichen Motiven eine deutlich heterogenere Mischung an Objekten als andere Benchmarks. Die gute Performanz auf diesem Datensatz lässt also hoffen, dass sich das DELF-Verfahren auch für die historische Domäne eignet.


