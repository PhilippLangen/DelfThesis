\section*{Motivation}
Mit der zunehmenden Digitalisierung unserer Datenwelt stehen jedem von uns heute mit wenigen Klicks mehr Informationen zur Verfügung, als wir je analog aufnehmen könnten. Auch wenn diese Flut an Informationen ein unglaubliches Potential für uns darstellt, so bürgt sie auch neue Herausforderungen. Um große Datenmengen sinnvoll nutzbar zu machen sind effiziente Suchwerkzeuge von zentraler Bedeutung. Dies gilt insbesondere auch für die Suche in großen Bilderdatenbanken. Bei der klassischen Bildersuche wird vom Nutzer eine Anfrage als String formuliert, auf welche das System eine Liste an Bildern mit groß möglichstem Bezug zur Anfrage liefert. Dabei werden häufig nützliche  Zusatzinformationen, sogenannte Metadaten der Bilder, wie Titel, Beschreibung, Aufnahmeort und Datum genutzt, um bessere Ergebnisse liefern zu können. Eine alternativer Ansatz der Suche ist die inhaltsbasierte Bildersuche (engl. Content-Based Image Retrieval kurz CBIR). Hierbei werden statt formulierten Anfragen Bilder als Anfragen gestellt. Ziel ist es Bilder mit gleichem oder ähnlichem Bildinhalt wie in der Anfrage als Ergebnis zurückzugeben. Dabei arbeitet das System direkt mit den Pixelinformationen der Bilder. Zusätzliche Metadaten sind daher nicht erforderlich. Im Folgenden wird die inhaltsbasierte Suche auch als Image Retrieval bezeichnet.
\\
Ein interessanter Anwendungsbereich für Image Retrieval Systeme ist die Suche auf historischen Bildern. Diese weit gefasste Domäne enthält sehr heterogene Daten. Dabei finden sich nicht nur sehr unterschiedliche Bildmotive, wie Gebäuden, Naturaufnahmen oder Portraits, sondern auch unterschiedliche Aufnahmetechniken bedingt durch den technischen Fortschritt über Zeichnungen, Malerei und Druck bis hin zur Photographie. Da Metadaten zu historischen Bildern erst bei der Digitalisierung hinzugefügt werden können sind diese oft gar nicht oder nur lückenhaft vorhanden, was die inhaltsbasierte Suche zu einem besonders geeigneten Ansatz macht. Mit der Umsetzung einer unterstützenden Suche für das UrbanHistory4D Projekt \cite{urbanhistory4d} ergibt sich ein konkreter Anwendungsfall für Image Retrieval Systeme. Hierbei handelt es sich um einen aktiven Forschungsbereich, in dem momentan unterschiedliche Suchsysteme analysiert werden. 
\\
Bei dem Image Retrieval Verfahren DELF (attentive DEep Local Features) \cite{delf}, welches in dieser Arbeit untersucht wird handelt es sich um einen Deep Learning Ansatz. Durch den raschen Fortschritt im Bereich tiefer neuronaler Netzwerkarchitekturen der letzten Jahre erfreuen sich gelernte Ansätze immer größerer Beliebtheit. DELF erzielt auf bekannten Benchmarkdatensätzen wie Oxford5k \cite{oxford5k} und Paris6k \cite{paris6k} sehr gute Ergebnisse. Besonders gut schneidet DELF im Vergleich auf dem eigens erstellten Google Landmarks Datensatz \cite{landmarks} ab. Dieser enthält mit über 1 mio. Bilder und 13k unterschiedlichen Motiven eine deutlich heterogener Mischung an Objekten. Die gute Performanz auf diesem Datensatz lässt als hoffen, dass sich das Verfahren auch für die historische Domäne eignet.


