\chapter{Verwandte Arbeiten}\label{related_work}
Bei Information Retrieval handelt es sich um ein Problem aus dem Bereich der Computer Vision, welches bereits seit langem intensiv erforscht wird. In frühen Ansätzen versuchte man vor allem globale Beschreibungen von Bildern zu erstellen, um diese untereinander vergleichen zu können. Diese basierten zum Beispiel auf Farbhistogrammen oder Texturbeschreibungen \cite{early_days}. Allerdings waren diese Ansätze oft sehr anfällig für Unterschiede in Beleuchtung, Skalierung und anderen Transformationen, wie sie bei unterschiedlichen Aufnahmen des selben Motivs auftreten können.
\\
Ein wesentlicher Durchbruch gelang David G. Lowe 2004 mit der Entwicklung des SIFT-Verfahrens (Scale Invariant Feature Transform) \cite{sift}. Hierbei werden mehrere Konzepte vereint um Bildbeschreibungen zu erzeugen, die robuster gegenüber unterschiedlichen Transformationen sind. So arbeitet der SIFT Algorithmus beispielsweise nicht direkt auf den Bildern, sondern im sogenannten Scale Space. Dieser besteht auf unterschiedlich skalierten Versionen des Ursprungsbildes, auf welche wiederum unterschiedlich starke Gauß-Filter angewendete werden. Betrachtet werden schließlich Differenzbilder zwischen benachbarten Stärken der Gauß-Fitler Ergebnisse. Die Verwendung von unterschiedlich skalierten Bildversionen macht die berechneten SIFT-Merkmale deutlich robuster gegen Skalierungsunterschiede. Das SIFT-Verfahren besteht aus zwei Phasen. In der ersten Phase werden über die Suche nach lokalen Extrema bedeutsame Bildpunkte ausgewählt. Für diese werden in der zweiten Phase einzelne Deskriptoren berechnet. Das Bild wird also nicht global beschrieben, sondern über viele lokale Deskriptoren dargestellt. Die lokalen Deskriptoren ergeben sich aus Histogrammen der Gradientrichtungen umliegender Bildpunkte. Diese werden relativ zu der dominanten Gradientrichtung in der Umgebung berechnet, was die Deskriptoren invariant gegenüber Rotationen macht. Lowes Entwicklung bildet den Ursprung für viele abgeleitete Verfahren wie SURF\cite{surf}, PCA-SIFT\cite{pca_sift} und RIFT\cite{rift}. Auch in aktueller Forschung werden Image Retrieval Verfahren untersucht, die mit SIFT-Merkmalen arbeiten \cite{modern_sift_useage}. \\
Der Trend bei der Entwicklung neuer Image Retrieval Systeme geht aktuell jedoch hauptsächlich in Richtung von gelernten Verfahren. Die Basis dieser Verfahren bilden tiefe CNN-Architekturen (Convolutional Neural Networks). Ein neuronales Netzwerk lässt sich als eine schichtweise Aneinanderreihung nicht-linearer Funktionen auffassen. Convolutional Neural Networks sind ein Sonderfall neuronaler Netze, welche sogenannte Convolutional Layer, zu deutsch faltende Schichten, enthalten. In diesen Schichten werden Faltungs/ bzw. Filteroperationen auf die Eingabedaten angewendet, um für das Netzwerk hilfreiche Merkmale in den Daten hervorzuheben. Dies ist durchaus vergleichbar mit den Filteroperationen, die im SIFT-Verfahren verwendet werden. Der Unterschied besteht jedoch darin, dass die Parameter der verwendeten Filtermasken sowie aller anderen Netzparameter nicht per Hand gewählt, sondern in einem Trainingsverfahren für den aktuellen Anwendungsfall optimiert werden. Der Entwickler bestimmt lediglich die grobe Architektur des Netzwerks, also die Anzahl, Größe und Reihenfolge der verwendeten Schichten sowie die Art der Operationen, die in ihnen durchgeführt werden. In Image Retrieval Systemen werden CNNs eingesetzt, um Bilddeskriptoren zu erstellen. Hierfür werden Zwischenergebnisse des Netzwerks, also die Ausgaben einer bestimmten Schicht genutzt. An welcher Stelle im Netzwerk die Deskriptoren entnommen werden ist dabei von entscheidender Bedeutung. Zeiler und Fergus haben in ihrer Studie zur Visualisierung von CNNs gezeigt \cite{extraction_point_meaning}, dass die früheren Schichten von CNNs typischerweise einfache Konzepte wie Kanten oder Ecken hervorheben. Mit wachsender Tiefe der betrachteten Netzwerkschicht steigt auch die Komplexität der Konzepte, die von den Ausgaben der Schicht beschrieben werden können.
\\\\
In dem in \cite{fc_extraction_neural_codes} beschriebenen Image Retrieval System von Babenko, Slesarev et al. wird als Modell ein CNN bestehend aus fünf faltenden gefolgt von drei voll verbundenen Schichten (im Englischen fully-connected layer) genutzt. Als Deskriptoren werden die Ausgaben der ersten bzw. zweiten voll verbundenen Schicht verwendet. In einer voll verbundenen Schicht hat jeder Wert der Eingabe Einfluss auf jeden Wert in der Ausgabe. Die Ausgaben solcher Schichten werden also von der gesamten Bildeingabe beeinflusst und können daher als globale Deskriptoren verstanden werden. Diese intuitive Herangehensweise erzielt leichte Verbesserung gegenüber den zur Zeit der Veröffentlichung gängigen algorithmischen Verfahren.
\\
Razavian, Sullivan et al. stellen in \cite{convnet} ein System auf Basis der in \cite{convnetarc} beschriebenen Netzwerkarchitektur vor. Das Modell besteht ebenfalls aus fünf faltenden und drei voll verbundenen Schichten. Die Deskriptoren stammen aus den Ausgaben der letzten faltenden Schicht.
%Die Dimensionalität der Deskriptoren wird durch Anwendung einer max-pooling Operation auf 2048 Dimensionen beschränkt.
Anders als bei Babenko, Slesarev et al. werden in diesem System mehrere Deskriptoren pro Bild erstellt. Hierfür werden systematisch Teilbilder aus Bildbereichen unterschiedlicher Größe generiert. Anschließend werden die Teilbilder auf eine feste Größe skaliert und als Eingabe in das Netzwerk gegeben. So wird für jeden betrachteten Bildbereich ein eigener lokaler Deskriptor erstellt. In ihren Experimenten stellen die Autoren fest, dass die Verwendung von lokalen Deskriptoren gegenüber einer globalen Betrachtung zu einer signifikanten Verbesserung der Retrievalperformanz führt. Der überwiegende Teil aktueller Retrieval Systeme setzt auf die Erstellung von lokalen Deskriptoren.
\\
Eine interessante Frage bei der Konzeption von Image Retrieval Systemen, die mit lokalen Deskriptoren arbeiten ist, wie man entscheidet, welche Bildregionen am sinnvollsten zu betrachten sind. Das ONE-Verfahren \cite{one} von Xie, Hong et al. nutzt ein VGG-19 \cite{vgg} Modell und extrahiert Deskriptoren aus der vorletzten voll verbundenen Schicht. Als Eingaben in das Netzwerk dienen sogenannte Object Proposals. Dabei handelt es sich um Bildausschnitte, welche Regionen umschließen, in denen Objekte vermutet werden. Die Autoren testen sowohl manuell annotierte sowie automatisch extrahierte Object Proposals und erzielen mit beiden Ansätzen ähnlich gute Ergebnisse. Für die automatische Bestimmung von Object Proposals nutzen sie das Selective Search Verfahren \cite{selective_search}.
\\
Das Delf-Verfahren \cite{delf}, welches in dieser Arbeit untersucht wird, basiert ebenfalls auf lokalen Deskriptoren. Die Deskriptoren werden aus einer faltenden Schicht aus dem hinteren Teil eines ResNet-50 \cite{resnet} Modells extrahiert. Als Eingabe in das Netzwerk werden Bilder in ihrer Gesamtheit betrachtet. Da bis zur Extraktionsschicht keine voll verbundenen Schichten genutzt werden, kann für jeden extrahierten Wert zurückgerechnet werden, von welchen Bereichen des Ursprungsbildes er beeinflusst wurde. Dies erlaubt es die Ausgaben der Extraktionsschicht in einzelne lokale Deskriptoren zu unterteilen. Um auszuwählen welche der lokalen Deskriptoren zur Darstellung eines Bildes genutzt werden sollen, werden die lokalen Deskriptoren in ein weiteres neuronales Netz gegeben. Dieses Netz hat die Aufgabe zu bewerten, wie geeignet die einzelnen Deskriptoren zur Beschreibung des Gesamtbildes sind. Auf Grund dieser Bewertung werden die wichtigsten Deskriptoren zu jedem Bild ausgewählt, wogegen schlecht bewertete Deskriptoren verworfen werden. Der konzeptionelle Unterschied bei der Auswahl der Deskriptoren im Vergleich zum ONE-Verfahren ist, dass die Auswahl auf Grund der bereits berechneten Deskriptoren geschieht anstatt auf Grund des Ursprungsbildes. Die Funktionsweise des Delf-Verfahrens wird in Kapitel \ref{delf_chapter} ab Seite \pageref{delf_chapter} im Detail erklärt.
\\  
Bevor Neuronale Netze für die Erstellung von Deskriptoren genutzt werden können, müssen ihre Parameter in einem Trainingsverfahren optimiert werden. Während dem Training muss das Netzwerk eine Aufgabe lösen. Wie erfolgreich das Netzwerk dabei ist, wird mit Hilfe einer Fehlerfunktion dargestellt. Das Netz versucht seine Parameter so anzupassen, dass die Fehlerfunktion minimiert wird. Im Fall der bereits vorgestellten Verfahren wird dabei eine Dummy-Aufgabe, typischerweise die Klassifikation von Bildern, gelöst. In der letzten Zeit wurden jedoch einige Ansätze veröffentlicht, die versuchen neuronale Netze direkt an Image Retrieval Aufgaben zu trainieren. Radenović, Tolias und Chum stellen in \cite{siamac_contrastive_loss} eine solchen Ansatz vor. Während dem Training betrachten sie dabei Bildpaare. Bei diesen Paaren handelt es sich entweder um korrekte Matches, falls die Bilder ähnliche Bildinhalte darstellen, oder um inkorrekte Matches, falls dies nicht der Fall ist. Die Bilder eines Paares durchlaufen identische Netze und erzeugen dabei jeweils eine Ausgabe. Für die Optimierung wird eine spezielle Fehlerfunktion definiert, welche bewertet wie stark sich die Netzwerkausgaben der Paare voneinander unterscheiden. Für korrekte Matches sollten sich die Netzwerkausgaben möglichst ähneln. Wird jedoch ein inkorrektes Match betrachtet, sollten auch die Ausgaben eine Mindestdifferenz zueinander aufweisen.
Die Autoren testen ihr Verfahren auf unterschiedlichen CNN Architekturen wie VGG \cite{vgg} und AlexNet \cite{alexnet} und erzielen damit sehr gute Ergebnisse auf gängigen Retrievalbenchmarks. Das direkte Training auf Retrievalaufgaben ist eine vielversprechende neue Forschungsrichtung im Retrievalbereich, an der momentan intensiv gearbeitet wird.
\\\\
% delf gesamtes bild als input durch conv layer wird auf lokale punkte zurückgerechnet. attention training auf den deskriptoren um gute auszuwählen. näheres  im konkreten kapitel
% diese system benötigen alle einen dummy task um die netzwerke zu trainieren oder benutzen vortrainierte netze alles mit klassifikation trainiert. es gibt aber ansätze die versuchen ein direkt für einen retrieval taks zu trainieren sia mac
Da Image Retrieval Systeme meist auf großen Bilddatenbanken eingesetzt werden und somit für eine Suchanfrage viele Bilder miteinander verglichen werden müssen, ist es sinnvoll Bildrepräsentationen so kompakt wie möglich zu gestalten, um die Laufzeit der Suche zu verbessern. Insbesondere bei Verfahren, die lokale Deskriptoren erstellen und häufig hunderte oder tausende Merkmale pro Bild extrahieren, kann mit einer guten Kodierung viel Rechenzeit gespart werden. Ein beliebter Ansatz zur Erstellung kompakter Darstellungen aus lokalen Deskriptoren ist das BOVW-Modell (Bag-of-Visual-Words) \cite{bow}, erstmals vorgestellt im Kontext von Textklassifikation von McCallum und Nigam. Hierbei werden zunächst alle aus einem Datensatz extrahierte Deskriptoren mittels Clusteranalyse (bspw. K-Means-Clustering \cite{k_means}) in Gruppen eingeteilt. Deskriptoren, die dem gleichen Cluster zugeordnet werden, werden dabei auf das selbe "visuelle Wort"\ abgebildet. Als Beschreibung des Gesamtbilds dient ein Histogramm über die im Bild enthaltenen visuellen Wörter. Bei diesem Verfahren geht durch Quantisierung ein Teil der Information verloren. Das ebenfalls auf Clustering basierte VLAD-Verfahren \cite{vlad} von Jégou, Douze et al. versucht diese Information nutzbar zu machen, indem es statt der Vorkommen die Quantisierungsfehler akkumuliert, die beim abbilden auf die nächsten visuellen Worte entstehen. \\
% ab hier könnte man auch aufhören
Um eine Suchanfrage mit einer Rangliste der ähnlichsten Bilder zum Suchbild beantworten zu können, werden die Deskriptoren der Bilder in der Datenbank mit denen des Suchbildes verglichen. Als Metrik dient hierbei meist die euklidische Distanz. Auf kleinen Datensätzen ist es laufzeittechnisch sinnvoll alle Kombinationen von Such- und Datenbankbildern zu vergleichen. Häufig werden bei größeren Datensätzen jedoch Methoden der approximierten nächsten Nachbarsuche (ANN) verwendet. Diese garantieren zwar kein optimales Ergebnis, erlauben jedoch deutlich schnellere Verarbeitung von Suchanfragen. So gibt es zum Beispiel Ansätze Deskriptoren mit Hilfe spezieller Hashfunktionen zu vergleichen. Diese werden so konstruiert, dass ähnliche Deskriptoren auf die gleichen bzw. möglichst ähnliche Hashcodes abgebildet werden, während gleichzeitig die Kollisionswahrscheinlichkeit für sehr unterschiedliche Deskriptoren minimal gehalten wird. Wang et al. beschreiben in ihrer Studie \cite{simsearch} unterschiedliche Konzepte für die Erstellung solcher Hashfunktionen.



%Den finale Teil einer Image Retrieval Pipeline bildet das Vergleichen der Anfragebilder mit den Bildern in der Suchdatenbank auf Basis der erzeugten Deskriptoren, mittels einer geeigneten Distanzmetrik (z.B. euklidische Distanz). Dabei ist ein erschöpfender Suchansatz auf Grund der Menge an Daten oft nicht sinnvoll. 




%Es gibt unterschiedliche Ansätze um das Problem der Suche nach den nächsten Nachbarn effizienter zu lösen. Ein Ansatz ist die Verwendung von räumlich partitionierenden Datenstrukturen, wie den von Friedman, Bentley und Finkel entwickelten K-D-Bäumen \cite{kd_tree}. Hierbei wird der Datenraum iterative entlang der unterschiedlichen Dimensionen der Elemente geteilt, wobei darauf geachtet wird, dass in den entstehenden Partitionen möglichst gleich viele Elemente enthalten sind. Dies wird solang wiederholt, bis in den einzelnen Partitionen nur noch eine geringe Anzahl an Elementen enthalten sind. Bei einer Suchanfrage 

