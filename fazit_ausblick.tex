\chapter{Fazit und Ausblick}

Ziel der vorliegenden Arbeit ist es zu untersuchen, ob das Image Retrieval System DELF für die inhaltsbasierte Suche auf historischen Daten geeignet ist und welche Stärken und Schwächen es im Vergleich zu alternativen Verfahren aufweist. Weiterhin sollen die Einflüsse unterschiedlicher Parameter innerhalb der DELF-Pipeline analysiert werden, um diese für die Domäne der historischen Aufnahmen zu optimieren.
\section{Fazit}
Es zeigt sich, dass das DELF-Verfahren bei der Anwendung auf historischen Aufnahmen nicht nur eine deutliche Verbesserung gegenüber einer zufälligen Suche darstellt, sondern in den meisten betrachteten Bilderkategorien auch signifikant bessere Ergebnisse als das im Vergleich betrachtete ConvNet-Verfahren erzielt. Trotz des relative hohen Rechenaufwands, der insbesondere durch den zusätzlichen Schritt der geometrischen Verifikation entsteht, ist DELF damit aktuell ein vielversprechendes Verfahren, für den Einsatz im HistStadt4D-Projekt.
\\
Eine Stärke des DELF-Verfahrens ist sein Umgang mit Luftaufnahmen, welche einen besonders schwierigen Aspekt des historischen Datensatzes bilden. So erzielt das DELF-Verfahren in Kategorien, die häufig in Luftaufnahmen zu finden sind, besonders deutliche Verbesserungen gegenüber dem ConvNet-Verfahren. 
\\
Als problematisch hat sich für das DELF-Verfahren der Umgang mit schlecht belichteten, bzw. Nachtaufnahmen erwiesen. So ist DELF kaum in der Lage gleiche Bildinhalte, bei sehr unterschiedlichen Belichtungsverhältnissen zu erkennen. Das ConvNet-Verfahren scheint hingegen kaum von dieser Problematik betroffen zu sein.
\\
Sowohl das DELF-, wie auch ConvNet-Verfahren können in Aufnahmen aus unterschiedliche zeitlichen Perioden, in denen sich die betrachteten Objekte durch Umbauten, oder Zerstörung stark voneinander unterscheiden, kaum gemeinsame Bildinhalte feststellen. Aktuell sind uns keine Möglichkeiten bekannt, mit solchen Veränderungen effektiv umzugehen.
\\
Objekte, die eine Vielzahl von sich ähnelnden Elementen wie Fenster oder Torbögen enthalten, sind für DELF, insbesondere wegen des geometrische Verifikationsschritts mittels RANSAC, problematisch. Diese sich wiederholenden Bildinhalte werden, auf Grund ihrer Ähnlichkeit zueinander, oft falschen Instanzen in dem betrachteten Vergleichsbildern zugeordnet, was zu geometrisch nicht erklärbaren Korrespondenzen führt. In der Parameteranalyse hat sich gezeigt, dass diese Problematik auf dem historischen Datensatz so ausgeprägt ist, dass das DELF-Verfahren hier ohne geometrische Verifikation gleich gute Ergebnisse erzielt. Die Vergleichsanalyse auf dem Benchmark-Datensatz Oxford5k zeigt sogleich, dass sich, mittels RANSAC, bei der Betrachtung von Objekten mit weniger Wiederholungen eine signifikante Verbesserung der Retrievalperformanz erreichen lässt. 
\\
Die Analyse unterschiedlicher Extraktionspunkte für die Erstellung von Deskriptoren hat ergeben, dass sich die letzten Schichten des ResNets nicht für die Extraktion von DELF-Deskriptoren eignen. Insbesondere das zur Selektion der Deskriptoren trainierte Attention-Netzwerk hat Schwierigkeiten geeignete Deskriptoren auszuwählen, wenn es auf den Ausgaben des letzten ResNet-Blocks trainiert wurde. Bei einer Extraktion aus dem vorletzten ResNet-Block, wie von den DELF-Autoren empfohlen, werden deutlich bessere Retrievalergebnisse erzielt.
\\
Ein weiterer betrachteter Parameter ist die verwendete Deskriptorlänge. Dabei zeigt sich, dass durch die Erstellung längerer Deskriptoren geringfügige Verbesserungen der Retrievalperformanz erzielt werden können. Dieser Effekt schwächt sich mit wachsender Länge der Deskriptoren deutlich ab. Zusätzlich steigt der benötigte Speicher- und Rechenbedarf mit der Länge der Deskriptoren an. Für einen geeigneten Kompromiss zwischen Retrievalperformanz und Speicherbedarf, bietet sich je nach verwendetem Datensatz eine Deskriptorlänge zwischen $40$ und $80$ Dimensionen an.
\\
Die zu Beginn durchgeführte Hyperparameteranalyse zur Optimierung der beiden Trainingsphasen des DELF-Verfahrens hat ergeben, dass die hier untersuchten Parameter in einem relativ großen Spektrum gewählt werden können, ohne die Trainingsergebnisse stark zu beeinflussen. Unter fast allen getesteten Konfigurationen, sowohl im Fine-Tuning, wie auch im Attention-Training, konnte das DELF-Netzwerk die gestellte Klassifikationsaufgabe mit sehr hoher Genauigkeit lösen. Es ist möglich, dass der relative kleine verwendete Trainingsdatensatz bestehend aus nur knapp vierzigtausend Bildern, keine ausreichende Herausforderung für die genutzte Modellarchitektur darstellt, um größere Unterschiede der Trainingsperformanz, auf Grund der Trainingsparameter, aufzuzeigen.

\section{Ausblick}
Da DELF unter den aktuell untersuchten Retrivalverfahren die besten Ergebnisse auf dem historischen Datensatz erzielt, ist es sinnvoll auf den Ergebnissen der vorliegenden Arbeit aufzubauen, um das Verfahren weiter für den Anwendungsfall im HistStadt4d-Projekt zu optimieren. Dabei gibt mehrere Bereiche, die sich für eine weitere Betrachtung anbieten.
\\
Eine Erweiterung der historischen Datenbestände ist für alle weiteren Untersuchungen zu Retrievalsystemen hilfreich. Aktuell umfasst der Datensatz nur $7$ unterschiedliche Kategorien, die teilweise mit sehr unterschiedlicher Häufigkeit vorkommen. Eine größere Vielfalt an Kategorien kann helfen, weitere Stärken und Schwächen der Verfahren zu erkennen und ermöglicht eine Untersuchung, die einem realistischen Anwendungsfall näher kommt. Ein ausgeglichenere Verteilung der unterschiedlichen Kategorien ermöglicht außerdem eine bessere Interpretation der Ergebnisse.
\\
Für das DELF-Training wäre auch eine Erweiterung der Trainingsdaten sehr nützlich. Das Aktuell verwendete Subset des Google Landmark V2 Datensatzes enthält einen großen Anteil an Gebäudeaufnahmen. Hierbei handelt es jedoch nur sehr selten um historische Aufnahmen. Auch Bildinhalte, die sich in den durchgeführt Experimenten als besonders herausfordernd gezeigt haben, wie Luft-, Nacht-, oder Aufnahmen von zerstörten Gebäuden finden sich kaum in den verwendeten Trainingsdaten. Das Hinzufügen von historischen Aufnahmen erlaubt es dem DELF-Verfahren möglicherweise besser mit solchen Inhalten umzugehen. Um geeignete Aufnahmen für einen Trainingssatz zu finden und zu annotieren können große Bilderarchive, wie Europeana\footnote{\url{https://www.europeana.eu/de}, zuletzt besucht 21.10.2020} genutzt werden, die eine Metadaten basierte Suche auf großen Datenmengen ermöglichen. Mit den aktuell bereits verfügbaren Retrievalsystemen können so gewonnene Suchergebnisse weiter verfeinert, bzw. Aufnahmen mit unerwünschten Inhalten aussortiert werden.
\\
Die Schwierigkeiten des DELF-Verfahrens Deskriptorpaare zwischen sich wiederholenden Bildinhalten geometrisch zu verifizieren sind eine der zentralen Erkenntnisse, die in der vorliegenden Arbeit gewonnen wurden. Somit stellt die Entwicklung einer Strategie, mit solchen Inhalten sinnvoll umzugehen und dabei weiterhin die Vorteile einer geometrischen Überprüfung zu nutzen eine wichtige Aufgabe der weiteren Forschungsarbeit zum DELF-Verfahren dar. Ein simpler Ansatz um diese Problematik zu lösen, wäre es für Deskriptoren, die große Ähnlichkeiten zu mehreren Deskriptoren in einem betrachteten Vergleichsbild haben, mehrere potentielle Korrespondenzen im RANSAC-Verfahrens zu betrachten. Somit steigt die Wahrscheinlichkeit das Deskriptorpaare zwischen den "korrekten" Instanzen sich wiederholender Elemente gefunden werden. 
\\
Im praktischen Einsatz spielt neben Qualität der Suchergebnisse auch die Geschwindigkeit des Suchsystems eine Rolle. Hier besteht aktuell noch viel Raum für weitere Optimierungen. Der für die vorliegenden Arbeit erstellte Prototyp des DELF-Verfahrens sammelt viele zusätzliche Informationen, die für die Verfahrensanalyse nützlich sind, die Laufzeit jedoch negativ beeinflussen. Um die Verarbeitung von Suchanfragen zu beschleunigen, ist es sinnvoll das DELF-Verfahren in einer schnelleren Programmiersprache wie C++ oder JAVA zu implementieren und effiziente Datenstrukturen, wie den in Kapitel \ref{pipeline_changes} vorgestellten invertierten Index, zu nutzen.
