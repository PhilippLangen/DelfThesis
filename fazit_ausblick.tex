\chapter{Fazit und Ausblick}

Ziel der vorliegenden Arbeit ist es zu untersuchen, ob das Image Retrieval System DELF für die inhaltsbasierte Suche auf historischen Daten geeignet ist und welche Stärken und Schwächen es im Vergleich zu alternativen Verfahren aufweist. Weiterhin sollen die Einflüsse unterschiedlicher Parameter innerhalb der DELF-Pipeline analysiert werden, um diese für die Domäne der historischen Aufnahmen zu optimieren.
\section{Fazit}
Es zeigt sich, dass das DELF-Verfahren bei der Anwendung auf historischen Aufnahmen nicht nur eine deutliche Verbesserung gegenüber einer zufälligen Suche darstellt, sondern in den meisten betrachteten Bilderkategorien auch signifikant bessere Ergebnisse als das Vergleichsverfahren ConvNet erzielt. Trotz des relative hohen Rechenaufwands, der insbesondere durch den zusätzlichen Schritt der geometrischen Verifikation entsteht, ist DELF damit aktuell ein vielversprechendes Verfahren, für den Einsatz im HistStadt4D-Projekt.
\\
Eine Stärke des DELF-Verfahrens ist sein Umgang mit Luftaufnahmen, welche einen besonders schwierigen Aspekt des historischen Datensatzes bilden. So zeigen sich in Kategorien die häufig in Luftaufnahmen gezeigt werden besonders große Performanzvorteile des DELF-Verfahrens gegenüber ConvNet.
\\
Als problematisch hat sich für das DELF-Verfahren der Umgang mit schlecht belichteten, bzw. Nachtaufnahmen erwiesen. So ist DELF kaum in der Lage gleiche Bildinhalte, bei sehr unterschiedlichen Belichtungsverhältnissen zu erkennen. Das ConvNet-Verfahren scheint hingegen kaum von dieser Problematik betroffen zu sein.
\\
Sowohl die DELF, wie auch ConvNet können in Aufnahmen aus unterschiedliche zeitlichen Perioden, in denen sich die betrachteten Objekte durch Umbauten, oder Zerstörung stark voneinander unterscheiden, kaum gemeinsame Bildinhalte feststellen. Aktuell sind uns keine Möglichkeiten bekannt, mit solchen Veränderungen effektiv umzugehen.
\\
Objekte, die eine Vielzahl von sich ähnelnden Elementen wie Fenster oder Torbögen enthalten, sind für DELF, insbesondere wegen des geometrische Verifikationsschritts mittels RANSAC, problematisch. Diese sich wiederholenden Bildinhalte werden, auf Grund ihrer Ähnlichkeit zueinander, oft falschen Instanzen in dem betrachteten Vergleichsbildern zugeordnet, was zu geometrisch nicht erklärbaren Korrespondenzen führt. In der Parameteranalyse hat sich gezeigt, dass diese Problematik auf dem historischen Datensatz so ausgeprägt ist, dass das DELF-Verfahren hier auch ohne geometrische Verifikation gleich gute Ergebnisse erzielt. Die Vergleichsanalyse auf dem Benchmark-Datensatz Oxford5k zeigt sogleich, dass sich, mittels RANSAC, bei der Betrachtung von Objekten mit weniger Wiederholungen eine signifikante Verbesserung der Retrievalperformanz erreichen lässt. 
\\
Die Analyse unterschiedlicher Extraktionspunkte für die Erstellung von Deskriptoren hat ergeben, dass sich die letzten Schichten des ResNets nicht für die Extraktion von DELF-Deskriptoren eignen. Insbesondere das zur Selektion der Deskriptoren trainierte Attention-Netzwerk hat Schwierigkeiten geeignete Deskriptoren auszuwählen, wenn es auf den Ausgaben des letzten ResNet-Blocks trainiert wurde. Bei einer Extraktion aus dem vorletzten ResNet-Block, wie von den DELF-Autoren empfohlen, werden deutlich bessere Retrievalergebnisse erzielt.
\\
Ein weiterer betrachteter Parameter ist die verwendete Deskriptorlänge. Dabei zeigt sich, dass durch die Erstellung längerer Deskriptoren geringfügige Verbesserungen der Retrievalperformanz erzielt werden können. Dieser Effekt schwächt sich mit wachsender Länge der Deskriptoren deutlich ab. Zusätzlich steigt der benötigte Speicher- und Rechenbedarf mit der Länge der Deskriptoren an. Für einen geeigneten Kompromiss zwischen Retrievalperformanz und Speicherbedarf, bietet sich je nach verwendetem Datensatz eine Deskriptorlänge zwischen $40$ und $80$ Dimensionen an.
\\
Die zu Beginn durchgeführte Hyperparameteranalyse zur Optimierung der beiden Trainingsphasen des DELF-Verfahrens hat ergeben, dass die hier untersuchten Parameter in einem relativ großen Spektrum gewählt werden können, ohne die Trainingsergebnisse stark zu beeinflussen. Unter fast allen getesteten Konfigurationen, sowohl im Fine-Tuning, wie auch im Attention-Training, konnte das DELF-Netzwerk die gestellte Klassifikationsaufgabe mit sehr hoher Genauigkeit lösen. Hierbei ist es möglich, dass der relative kleine verwendete Trainingsdatensatz bestehend aus nur knapp vierzigtausend Bildern, keine ausreichende Herausforderung für die genutzte Modellarchitektur darstellt, um größere Unterschiede der Trainingsperformanz, auf Grund der Trainingsparameter, aufzuzeigen.

\section{Ausblick}
