\documentclass[plainarticle,zihtitle,german,final,hyperref,utf8]{zihpub}
\usepackage{setspace}
\author{Guido Juckeland}
\title{Die ZIH-\LaTeX-Formatvorlage}
\bibfiles{doku}

\birthday{1. Januar 1970}
\placeofbirth{Musterstadt}
\betreuer{Dr. Max Mustermann}

\begin{document}

\section{Aller Anfang ist schwer}
Gr\"o{\ss}ere Ausarbeitungen an der Professur f\"ur Rechnerarchitektur bzw. am Zentrum f\"ur Informationsdienste und Hochleistungsrechnen sollen, um ein einheitliches Layout zu erhalten, mittels {\LaTeX} und einer gemeinsamen Formatvorlage erfolgen. 
Dieses Dokument soll den Einstieg in das Erstellen von Dokumenten mit {\LaTeX} erleichtern sowie die Verwendung dieser Formatvorlage erl\"autern.

{\LaTeX} ist eine Dokumentbeschreibungssprache und in dieser Hinsicht HTML nicht un\"ahnlich. 
Es wurde entwickelt um einfach gut aussehende (vor allem wissenschaftliche) Ver\"offentlichungen zu erm\"oglichen. 
F\"ur Grundlagen zu Syntax und Aufbau von {\LaTeX}-Dokumenten sei an dieser Stelle auf zwei hervorragende Einf\"uhrungen, und gleichzeitig Referenzen, verwiesen: \cite{kochbuch} sowie \cite{rudl}. 
Weiterhin ist zu {\LaTeX} auch einschl\"agige Literatur vorhanden -- genannt werden sollen an dieser Stelle exemplarisch \cite{begleiter} und \cite{kopka}. 
Im Folgenden soll im Besonderen auf den Umgang mit der hier zur Verf\"ugung gestellten Formatvorlage eingegangen werden.

%%%%%%%%%%%%%%%%%%%%%%%%%%%%%%%%%%%%%%%%%%%%%%%%%%%%%%%%%%%%%%%
\section{Die ZIH-Formatvorlage}
Die Formatvorlage ist sowohl f\"ur die Erstellung wissenschaftlicher Arbeiten als auch zur Erzeugung von ZIH-Reports geeignet. 
Sie besteht im Moment aus den Dateien \texttt{zihpub.cls}, \texttt{alphadin.bst}, \texttt{plaindin.bst}, \texttt{Makefile} und dieser Dokumentation (\texttt{doku.pdf}).


\subsection{Einbinden der Vorlage}
Am einfachsten nutzt sich die Vorlage, in dem alle 3 Dateien (\texttt{zihpub.cls, alphadin.bst, plaindin.bst}) im selben Verzeichnis wie die {\LaTeX}-Dokumente der Seminararbeit platziert werden. 
{\LaTeX} pr\"uft beim Erstellen des Dokumentes zuerst das aktuelle Verzeichnis auf das Vorhandensein der ben\"otigten Vorlagen.

\Paragraph{Ben\"otigte {\LaTeX}-Pakete}
Diese Formatvorlage ben\"otigt die folgenden Pakete, die jedoch in jeder Standardinstallation enthalten sind:
\begin{itemize}
	\item  Koma-Script (komplett),
	\item \texttt{inputenc} und \texttt{fontenc},
	\item \texttt{setspace}
	\item \texttt{times},
	\item \texttt{graphicx},
	\item \texttt{tabularx},
	\item \texttt{longtable},
	\item \texttt{url},
	\item \texttt{color},
	\item \texttt{enumerate},
	\item \texttt{hyperref},
    	\item \texttt{babel-german} und \texttt{babel-english},
	\item \texttt{fancyvrb} und
	\item \texttt{amsmath}, \texttt{amsfonts} und \texttt{amssymb}.
\end{itemize}

%%%%%%%%%%%%%%%%%%%%%%%%%%%%%%%%%%%%%%%%%%%%%%%%%%%%%%%%%%%%%%%%%%%%%%%%
\subsection{Verwenden der Vorlage}

\subsubsection{Seminararbeiten und ZIH-Artikel}
Die Vorlage f\"ur Seminararbeiten und ZIH-Artikel basiert auf dem Dokumenttyp \texttt{scrartcl} aus dem Koma-Script-Paket. 
Die h\"ochste Gliederungsebene ist daher \verb|\section{...}| (gefolgt von \\\verb|\subsection{...}|, \verb|\subsubsection{...}|, \verb|\paragraph{...}| und \\\verb|\subparagraph{...}|). 
Das {\LaTeX}-Ger\"ust einer Seminararbeit, die die ZIH-Vorlage benutzt, sollte ungef\"ahr wie folgt aussehen:
\begin{Verbatim}[baselinestretch=1,fontsize=\scriptsize,numbers=left,frame=single,stepnumber=5,xleftmargin=1cm,xrightmargin=1cm]
\documentclass[german,proseminar]{zihpub}

\author{Guido Juckeland}
\title{Die ZIH-{\LaTeX}-Formatvorlage}
\matno{1234567}
\betreuer{Dr. Max Mustermann}
\bibfiles{bib-filenames}
\copyrighterklaerung{Hier soll jeder Autor die von ihm eingeholten
Zustimmungen der Copyright-Besitzer angeben bzw. die in Web Press
Rooms angegebenen generellen Konditionen seiner Text- und
Bild\"ubernahmen zitieren.}
\acknowledgments{Die Danksagung...}

\begin{document}

 % Hier kommt die Seminararbeit hin...

\end{document}
\end{Verbatim}

Dabei sind nat\"urlich die Felder \textit{author}, \textit{title}, \textit{matno} (Matrikelnummer), \textit{betreuer}, \textit{bibfiles} und \textit{co\-py\-right\-er\-klae\-rung} entsprechend an die eigene Arbeit anzupassen. 
Sollte keine Literatur verwendet werden, kann \textit{bibfiles} weggelassen werden (dasselbe gilt f\"ur \textit{copyrighterklaerung}). 
Soll als Dokumentdatum (auf der Titelseite) nicht das heutige Datum gew\"ahlt werden, so kann mittels \verb|\date{neues Datum}| ein anderes Datum gesetzt werden. 
Bei Arbeiten, die in Kooperation mit anderen Einrichtungen entstanden sind, ist es \"ublich eine Danksagung an das Dokument anzuf\"ugen. 
Dies kann mit \textit{acknowledgments} erfolgen.

Aus folgenden Dokumentoptionen ist EINE als Dokumentyp auszuw\"ahlen und in die eckigen Klammern nach \verb|\documentclass| zu setzen:
\begin{itemize}
	\item \texttt{proseminar}: F\"ur einen Beitrag zum Proseminar ``Rechnerarchitektur''
	\item \texttt{hauptseminar}: F\"ur einen Beitrag zum Hauptseminar ``Rechnerarchitektur und Programmierung''
	\item \texttt{mathseminar}: F\"ur einen Beitrag zum Seminar ``Programmier- und Compilertechniken im Wissenschaftlichen Hochleistungsrechnen''
	\item \texttt{plainarticle}: F\"ur eine neutrale Ausarbeitung, die auf dem Grunddokumenttyp \texttt{article} aufbaut (f\"ur ZIH-Artikel). Die Felder \texttt{matno} und \texttt{betreuer} sind dann ohne Bedeutung.
\end{itemize}

Folgende weiteren Dokumentoptionen stehen zur Verf\"ugung:
\begin{itemize}
  \item \texttt{bibnum}: Literaturverweise nur mit einer Nummer (z.B. [1]) und nicht alphanumerisch (z.B. [GuJu04]).
  \item \texttt{final}: \"Andert die Formatierungen im Dokument von den Vorgaben f\"ur wissenschaftliche Arbeiten zu den Vorgaben zur Ver\"offentlichung als ZIH-(Internal)-Report. Dabei ist zu beachten, dass sich der Zeilenabstand ver\"andert. Seitenumbr\"uche und Positionierung von Grafiken/Tabellen sind danach evtl. neu zu gestalten. Weiterhin ist das Dokument dann zweiseitig.
	\item \texttt{german}: F\"ur ein deutsches Dokument. (Standard: Englisch)
  \item \texttt{hyperref}: Aktiviert die Hyperref-Unterst\"utzung f\"ur das Dokument. Erlaubt das Navigieren durch Klicken auf Referenzen.
  \item \texttt{lof}: Erzeugt ein Abbildungsverzeichnis.
  \item \texttt{lot}: Erzeugt ein Tabellenverzeichnis.
  \item \texttt{nomencl}: F\"ugt dem Dokument ein Symbolverzeichnis hinzu. Weitere Informationen dazu im Abschnitt \ref{sec:nomencl}.
  \item \texttt{notoc}: Entfernt das Inhaltsverzeichnis aus dem Dokument.
  \item \texttt{notitlepage}: Entfernt das Titelblatt aus dem Dokument. Zur Verwendung eigener Titelseiten, z.B. mittels \verb|\includepdf| oder\\ \verb|\AtBeginDocument{ \begin{titlepage} ... \end{titlepage} }|.
  \item \texttt{zihtitle}: Erzeugt ein Deckblatt mit den Angaben des ZIH anstelle der Professur f\"ur Rechnerarchitektur bzw. der Professur f\"ur Angewandte Diskrete Mathematik.
  \item \texttt{twoside}: Erzeugt ein zweiseitiges Dokument (f\"ur gro{\ss}e Arbeiten).
  \item \texttt{utf8}: Verwendet UTF-8 Encoding (Standard: Latin-1).
\end{itemize}

%%%%%%%%%%%%%%%%%%%%%%%%%%%%%%%%%%%%%%%%%%%%%%%%%%%%%%%%%
\subsubsection{Gro{\ss}e Belege und ZIH-Berichte}

Die Vorlage f\"ur gro{\ss}e Belege und ZIH-Berichte basiert auf dem Dokumenttyp \texttt{scrreprt} aus dem Koma-Script-Paket. 
Die h\"ochste Gliederungsebene ist daher \verb|\chapter{...}| (gefolgt von \\\verb|\section{...}|, \verb|\subsection{...}|, \verb|\subsubsection{...}|, \verb|\paragraph{...}| und \\\verb|\subparagraph{...}|). 
Das {\LaTeX}-Ger\"ust eines gro{\ss}en Belegs, der die ZIH-Vorlage benutzt, sollte ungef\"ahr wie folgt aussehen:
\begin{Verbatim}[baselinestretch=1,fontsize=\scriptsize,numbers=left,frame=single,stepnumber=5,xleftmargin=1cm,xrightmargin=1cm]
\documentclass[german,beleg]{zihpub}

\author{Guido Juckeland}
\title{Die ZIH-{\LaTeX}-Formatvorlage}
\matno{1234567}
\betreuer{Dr. Max Mustermann}
\bibfiles{bib-filenames}
\copyrighterklaerung{Hier soll jeder Autor die von ihm eingeholten
Zustimmungen der Copyright-Besitzer angeben bzw. die in Web Press
Rooms angegebenen generellen Konditionen seiner Text- und
Bild\"ubernahmen zitieren.}
\acknowledgments{Die Danksagung...}

\begin{document}

 % Hier kommt die Belegarbeit hin...

\end{document}
\end{Verbatim}

Dabei sind nat\"urlich die Felder \textit{author}, \textit{title}, \textit{matno} (Matrikelnummer), \textit{betreuer}, \textit{bibfiles} und \textit{copyrighterklaerung} entsprechend an die eigene Arbeit anzupassen. 
Sollte keine Literatur verwendet werden, kann \textit{bibfiles} weggelassen werden (dasselbe gilt f\"ur \textit{copyrighterklaerung}). 
Soll als Dokumentdatum (auf der Titelseite) nicht das heutige Datum gew\"ahlt werden, so kann mittels \verb|\date{neues Datum}| ein anderes Datum gesetzt werden. 
Bei Arbeiten, die in Kooperation mit anderen Einrichtungen entstanden sind, ist es \"ublich eine Danksagung an das Dokument anzuf\"ugen. 
Dies kann mit \textit{acknowledgments} erfolgen.

Aus folgenden Dokumentoptionen ist EINE als Dokumenttyp auszuw\"ahlen und in die eckigen Klammern nach \verb|\documentclass| zu setzen:
\begin{itemize}
	\item \texttt{beleg}: F\"ur einen gro{\ss}en Beleg
	\item \texttt{plainreport}: F\"ur eine neutrale Ausarbeitung, die auf dem Grunddokumenttyp \texttt{report} aufbaut (f\"ur ZIH-Berichte). Die Felder \texttt{matno} und \texttt{betreuer} sind dann ohne Bedeutung.
\end{itemize}

Folgende weiteren Dokumentoptionen stehen zur Verf\"ugung:
\begin{itemize}
  \item \texttt{bibnum}: Literaturverweise nur mit einer Nummer (z.B. [1]) und nicht alphanumerisch (z.B. [GuJu04]).
  \item \texttt{final}: Ändert die Formatierungen im Dokument von den Vorgaben f\"ur wissenschaftliche Arbeiten zu den Vorgaben zur Ver\"offentlichung als ZIH-(Internal)-Report. Dabei ist zu beachten, dass sich der Zeilenabstand ver\"andert. Seitenumbr\"uche und Positionierung von Grafiken/Tabellen sind danach evtl. neu zu gestalten. Weiterhin ist das Dokument dann zweiseitig.
	\item \texttt{german}: F\"ur ein deutsches Dokument. (Standard: Englisch)
  \item \texttt{hyperref}: Aktiviert die Hyperref-Unterst\"utzung f\"ur das Dokument. Erlaubt das Navigieren durch Klicken auf Referenzen.
  \item \texttt{lof}: Erzeugt ein Abbildungsverzeichnis.
  \item \texttt{lot}: Erzeugt ein Tabellenverzeichnis.
  \item \texttt{nomencl}: F\"ugt dem Dokument ein Symbolverzeichnis hinzu. Weitere Informationen dazu im Abschnitt \ref{sec:nomencl}.
    \item \texttt{notoc}: Entfernt das Inhaltsverzeichnis aus dem Dokument.
	\item \texttt{zihtitle}: Erzeugt ein Deckblatt mit den Angaben des ZIH anstelle der Professur f\"ur Rechner\-architektur.
	 \item \texttt{notitlepage}: Entfernt das Titelblatt aus dem Dokument. Zur Verwendung eigener Titelseiten, z.B. mittels \verb|\includepdf| oder\\ \verb|\AtBeginDocument{ \begin{titlepage} ... \end{titlepage} }|.
	\item \texttt{twoside}: Erzeugt ein zweiseitiges Dokument (f\"ur gro{\ss}e Arbeiten).
	\item \texttt{utf8}: Verwendet UTF-8 Encoding (Standard: Latin-1).
\end{itemize}

%%%%%%%%%%%%%%%%%%%%%%%%%%%%%%%%%%%%%%%%%%%%%%%%%%%%%%%%%
\subsubsection{Diplomarbeiten, Bachelor-, Master-Arbeiten}

Die Vorlage f\"ur Diplomarbeiten, Bachelor-, Master-Arbeiten basiert auf dem Dokumenttyp \texttt{scrreprt} aus dem Koma-Script-Paket. 
Die h\"ochste Gliederungsebene ist daher \verb|\chapter{...}| (gefolgt von \verb|\section{...}|, \\\verb|\subsection{...}|, \verb|\subsubsection{...}|, \verb|\paragraph{...}| und \\\verb|\subparagraph{...}|). 
Das {\LaTeX}-Ger\"ust einer Diplom-, Bachelor-, Master-Arbeit, das die ZIH-Vorlage benutzt, sollte ungef\"ahr wie folgt aussehen:
\begin{Verbatim}[baselinestretch=1,fontsize=\scriptsize,numbers=left,frame=single,stepnumber=5,xleftmargin=1cm,xrightmargin=1cm]
\documentclass[diplomist,german]{zihpub}

\author{Guido Juckeland}
\title{Die ZIH-{\LaTeX}-Formatvorlage}
\birthday{1. Januar 1970}
\placeofbirth{Musterstadt}
\betreuer{Dr. Max Mustermann}
\bibfiles{bib-filenames}
\copyrighterklaerung{Hier soll jeder Autor die von ihm eingeholten
Zustimmungen der Copyright-Besitzer angeben bzw. die in Web Press
Rooms angegebenen generellen Konditionen seiner Text- und
Bild\"ubernahmen zitieren.}
\acknowledgments{Die Danksagung...}
\abstractde{Abstract in Deutsch}
\abstracten{Abstract in Englisch}

\begin{document}

 % Hier kommt die Diplomarbeit hin...

\end{document}
\end{Verbatim}

Dabei sind nat\"urlich die Felder \textit{author}, \textit{title}, \textit{birthday}, \textit{placeofbirth}, \textit{betreuer}, \textit{bibfiles}, \textit{copyrighterklaerung}, \textit{abstractde} und \textit{abstracten} entsprechend an die eigene Arbeit anzupassen. 
Das Geburtsdatum ist dabei in der Form \textit{01. Januar 1970} anzugeben. 
Sollte keine Literatur verwendet werden, kann \textit{bibfiles} weggelassen werden (dasselbe gilt f\"ur \textit{copyrighterklaerung}). 
Soll als Dokumentdatum (auf der Titelseite) nicht das heutige Datum gew\"ahlt werden, so kann mittels \verb|\date{neues Datum}| ein anderes Datum gesetzt werden. 
Bei Diplomarbeiten oder Arbeiten, die in Kooperation mit anderen Einrichtungen entstanden sind, ist es \"ublich eine Danksagung an das Dokument anzuf\"ugen. 
Dies kann mit \textit{acknowledgments} erfolgen. 
Diplomandinnen verwenden bitte zus\"atzlich die Dokumentoption \textit{female}.
Wird die Arbeit von einem anderen Hochschullehrer betreut, so kann er mit dem Befehl \textit{hsl} angegeben werden.

Aus folgenden Dokumentoptionen ist EINE als Dokumentyp auszuw\"ahlen und in die eckigen Klammern nach \verb|\documentclass| zu setzen:
\begin{itemize}
	\item \texttt{diplominf}: F\"ur eine Diplomarbeit zur Erlangung des akademischen Grades Diplom-Infor\-ma\-ti\-ker(in)
	\item \texttt{bachinf}: F\"ur eine Bachelor-Arbeit zur Erlangung des akademischen Grades Bachelor of Science
	\item \texttt{mastinf}: F\"ur eine Master-Arbeit zur Erlangung des akademischen Grades Master of Science
	\item \texttt{diplomist}: F\"ur eine Diplomarbeit zur Erlangung des akademischen Grades Diplom-In\-ge\-nieur (-in) f\"ur Infor\-ma\-tions\-sys\-tem\-tech\-nik
	\item \texttt{diplomtmath}: F\"ur eine Diplomarbeit zur Erlangung des akademischen Grades Diplom-Ma\-the\-ma\-ti\-ker(in) (Technomathematik)
\end{itemize}

Folgende weiteren Dokumentoptionen stehen zur Verf\"ugung:
\begin{itemize}
  \item \texttt{bibnum}: Literaturverweise nur mit einer Nummer (z.B. [1]) und nicht alphanumerisch (z.B. [GuJu04])
  \item \texttt{female}: F\"ugt bei Diplomarbeiten das ``in'' an den Titel an.
  \item \texttt{final}: \"Andert die Formatierungen im Dokument von den Vorgaben f\"ur wissenschaftliche Arbeiten zu den Vorgaben zur Ver\"offentlichung als ZIH-(Internal)-Report. Dabei ist zu beachten, dass sich der Zeilenabstand ver\"andert. Seitenumbr\"uche und Positionierung von Grafiken/Tabellen sind danach evtl. neu zu gestalten. Weiterhin ist das Dokument dann zweiseitig.
  \item \texttt{german}: F\"ur ein deutsches Dokument. (Standard: Englisch)
  \item \texttt{hyperref}: Aktiviert die Hyperref-Unterst\"utzung f\"ur das Dokument. Erlaubt das Navigieren durch klicken auf Referenzen.
  \item \texttt{lof}: Erzeugt ein Abbildungsverzeichnis.
  \item \texttt{lot}: Erzeugt ein Tabellenverzeichnis.
  \item \texttt{nomencl}: F\"ugt dem Dokument ein Symbolverzeichnis hinzu. Weitere Informationen dazu im Abschnitt \ref{sec:nomencl}.
  \item \texttt{noproblem}: Entfernt den Platzhalter f\"ur die Aufgabenstellung.
  \item \texttt{notoc}: Entfernt das Inhaltsverzeichnis aus dem Dokument.
   \item \texttt{notitlepage}: Entfernt das Titelblatt aus dem Dokument. Zur Verwendung eigener Titelseiten, z.B. mittels \verb|\includepdf| oder\\ \verb|\AtBeginDocument{ \begin{titlepage} ... \end{titlepage} }|.
  \item \texttt{zihtitle}: Erzeugt ein Deckblatt mit den Angaben des ZIH anstelle der Professur f\"ur Rechnerarchitektur.
  \item \texttt{twoside}: Erzeugt ein zweiseitiges Dokument (f\"ur gro{\ss}e Arbeiten).
  \item \texttt{utf8}: Verwendet UTF-8 Encoding (Standard: Latin-1).
\end{itemize}




%%%%%%%%%%%%%%%%%%%%%%%%%%%%%%%%%%%%%%%%%%%%%%%%%%%%%%%%%
\subsubsection{Dissertationen}

Die Vorlage f\"ur Dissertationen basiert auf dem Dokumenttyp \texttt{scrreprt} aus dem Koma-Script-Paket. 
Die h\"ochste Gliederungsebene ist daher \verb|\chapter{...}| (gefolgt von \verb|\section{...}|, \\\verb|\subsection{...}|, \verb|\subsubsection{...}|, \verb|\paragraph{...}| und \\\verb|\subparagraph{...}|). 
Das {\LaTeX}-Ger\"ust einer Disseration, das die ZIH-Vorlage benutzt, sollte ungef\"ahr wie folgt aussehen:
\begin{Verbatim}[baselinestretch=1,fontsize=\scriptsize,numbers=left,frame=single,stepnumber=5,xleftmargin=1cm,xrightmargin=1cm]
\documentclass[dissrernat,utf8,hyperred,twoside]{zihpub}

\author{Guido Juckeland}
\title{Die ZIH-{\LaTeX}-Formatvorlage}
\birthday{1. Januar 1970}
\placeofbirth{Musterstadt}
\gutachter{Prof. Dr. rer. nat. Wolgang E. Nagel}
\bibfiles{bib-filenames}

\begin{document}

 % Hier kommt die Dissertation hin...

\end{document}
\end{Verbatim}

Dabei sind nat\"urlich die Felder \textit{author}, \textit{title}, \textit{birthday}, \textit{placeofbirth}, \textit{gutachter}, \textit{bibfiles} entsprechend an die eigene Arbeit anzupassen. 
Das Geburtsdatum ist dabei in der Form \textit{01. Januar 1970} anzugeben. 
Sollte keine Literatur verwendet werden, kann \textit{bibfiles} weggelassen werden (dasselbe gilt f\"ur \textit{copyrighterklaerung}). 
Soll als Dokumentdatum (auf der Titelseite) nicht das heutige Datum gew\"ahlt werden, so kann mittels \verb|\date{neues Datum}| ein anderes Datum gesetzt werden. 
Bei Dissertationen oder Arbeiten, die in Kooperation mit anderen Einrichtungen entstanden sind, ist es \"ublich eine Danksagung an das Dokument anzuf\"ugen. 
Dies kann mit \textit{acknowledgments} erfolgen. Eine Widmung kann mit \textit{dedication} hinzugef\"ugt werden. 
Ebenso k\"onnen die Abstracts genau wie bei Diplomarbeiten eingef\"ugt werden.

Aus folgenden Dokumentoptionen ist EINE als Dokumentyp auszuw\"ahlen und in die eckigen Klammern nach \verb|\documentclass| zu setzen:
\begin{itemize}
	\item \texttt{dissrernat}: F\"ur eine Dissertation zur Erlangung des akademischen Grades Doktor rerum naturalium (Dr. rer. nat.)
        \item \texttt{dissing}: F\"ur eine Dissertation zur Erlangung des akademischen Grades Doktoringenieur (Dr.-Ing.)
\end{itemize}

Folgende weiteren Dokumentoptionen stehen zur Verf\"ugung:
\begin{itemize}
  \item \texttt{bibnum}: Literaturverweise nur mit einer Nummer (z.B. [1]) und nicht alphanumerisch (z.B. [GuJu04])
  \item \texttt{final}: \"andert die Formatierungen im Dokument von den Vorgaben f\"ur wissenschaftliche Arbeiten zu den Vorgaben zur Ver\"offentlichung als ZIH-(Internal)-Report. Dabei ist zu beachten, dass sich der Zeilenabstand ver\"andert. Seitenumbr\"uche und Positionierung von Grafiken/Tabellen sind danach evtl. neu zu gestalten. Weiterhin ist das Dokument dann zweiseitig.
  \item \texttt{kurz}: Erzeugt die Kurzfassung
  \item \texttt{german}: F\"ur ein deutsches Dokument. (Standard: Englisch)
  \item \texttt{hyperref}: Aktiviert die Hyperref-Unterst\"utzung f\"ur das Dokument. Erlaubt das Navigieren durch klicken auf Referenzen.
  \item \texttt{lof}: Erzeugt ein Abbildungsverzeichnis.
  \item \texttt{lot}: Erzeugt ein Tabellenverzeichnis.
  \item \texttt{nomencl}: F\"ugt dem Dokument ein Symbolverzeichnis hinzu. Weitere Informationen dazu im Abschnitt \ref{sec:nomencl}.
  \item \texttt{notoc}: Entfernt das Inhaltsverzeichnis aus dem Dokument.
   \item \texttt{notitlepage}: Entfernt das Titelblatt aus dem Dokument. Zur Verwendung eigener Titelseiten, z.B. mittels \verb|\includepdf| oder\\ \verb|\AtBeginDocument{ \begin{titlepage} ... \end{titlepage} }|.
  \item \texttt{twoside}: Erzeugt ein zweiseitiges Dokument (f\"ur gro{\ss}e Arbeiten).
  \item \texttt{utf8}: Verwendet UTF-8 Encoding (Standard: Latin-1).
\end{itemize}

\subsection{Zur Verf\"ugung gestellte Befehle}
Innerhalb der Formatvorlage werden h\"aufig ben\"otigten Pakete eingebunden. 
Zur Verwendung der von den Paketen zur Verf\"ugung gestellten Befehle sei auf deren Dokumentation verwiesen (meist in <TeX-root>/doc/latex zu finden). Dies sind:

\begin{itemize}
	\item \texttt{inputenc}: Erlaubt die direkte Verwendung von Umlauten und deutschen Sonderzeichen innerhalb des Quellcodes. Siehe dazu auch Abschnitt \ref{sec:umlaute}.
	\item \texttt{fontenc}: Erm\"oglicht die automatische Silbentrennung von Worten mit Umlauten.
	\item \texttt{times}: Setzt die Dokumentschriftart auf Adobe Times.
	\item \texttt{graphicx}: Erm\"oglicht die Einbindung von Grafiken. Siehe dazu auch Abschnitt \ref{sec:grafiken}.
	\item \texttt{tabularx}: Erm\"oglicht die Erstellung von Tabellen mit Spalten gleicher Breite. Zur Verwendung sei auf die zum Paket geh\"orige Dokumentation verwiesen.
	\item \texttt{longtable}: Erm\"oglicht die Erstellung von Tabellen mit einem Seitenumbruch innerhalb der Tabelle. Zur Verwendung sei auf die zum Paket geh\"orige Dokumentation verwiesen.
	\item \texttt{url}: Erm\"oglicht das Verwenden von URL's mittels \verb|\url\{URL\}|.
	\item \texttt{color}: Erm\"oglicht farbigen Text/Texthintergr\"unde.
    \item \texttt{babel}: Sprachdefinitionen f\"ur Deutsch und Englisch.
	\item \texttt{fancyvrb}: Erm\"oglicht schicke Listings (wie in diesem Dokument verwendet). Zur Verwendung sei auf die Dokumentation zum Paket unter \texttt{.../textmf/doc/latex/fancyvrb} verwiesen.
	\item \texttt{amsmath}, \texttt{amsfonts} und \texttt{amssymb}: Stellen die {\LaTeX}-Mathematik-Formatierungen und -Sym\-bo\-le zur Verf\"ugung.
	\item \texttt{enumerate}: Erlaubt nummerierte Aufz\"ahlungen mit benutzerdefinierten Aufz\"ahlungszeichen.
	\item \texttt{nomencl}: Erzeugt \"uber die makeindex-Umgebung ein Symbolverzeichnis.
\end{itemize}

\noindent Folgende benutzerdefinierten Befehle stehen zur Verwendung bereit:
\begin{itemize}
	\item \verb|\longcite{Zitierschl\"ussel}{Seitenzahl}| erzeugt einen Eintrag f\"ur das Verweisen auf ein Dokument, aus dem die dargestellten Fakten \"ubernommen wurden in folgender Form: \longcite{rudl}{123}.
	\item \verb|\shortcite{Zitierschl\"ussel}{Seitenzahl}| erzeugt einen Eintrag, f\"ur das direkte Zitieren einer Quelle in folgender Form: \shortcite{rudl}{123}.
	\item \verb|\Paragraph{Überschrift}|: Erzeugt eine Absatz\"uberschrift mit darauffolgendem Zeilen\-umbruch.
	\item \verb|\correctme{Text}|: F\"arbt \texttt{Text} rot ein, um ihn als ``Noch zu bearbeiten'' zu kennzeichnen. Dies funktioniert jedoch nur f\"ur maximal einen Absatz.
	\item \verb|\begin{correctmore} Text \end{correctmore}|: F\"arbt \texttt{Text} rot ein, um ihn als ``Noch zu bearbeiten'' zu kennzeichnen. Dies funktioniert auch \"uber Absatzgrenzen hinweg.
\end{itemize}

%%%%%%%%%%%%%%%%%%%%%%%%%%%%%%%%%%%%%%%%%%%%%%%%%%%%%%%%%%%%%%%%%%%
\section{Besonderheiten und Hinweise}

\subsection{Grafiken und Tabellen}
\label{sec:grafiken}

Grafiken platziert man am besten wie folgt (was zu dem in Abbildung \ref{fig:zhr_logo} dargestellten Ergebnis f\"uhrt):
\begin{Verbatim}[baselinestretch=1,fontsize=\scriptsize,numbers=left,frame=single,stepnumber=5,xleftmargin=1cm,xrightmargin=1cm]
\begin{figure}[htbp]
  \begin{center}
    \includegraphics[width=0.5\textwidth]{Logo_20mm_sw.pdf}
    \caption{Das ZIH-Logo}
    \label{fig:bild1}
  \end{center}
\end{figure}
\end{Verbatim}
Dabei kann die Bildbreite relativ zur Textbreite auch variiert werden und zwar mittels der Option\\ \verb|width=0.75\textwidth|.\\
Die Bildunterschrift ist unter der Grafik zu platzieren (wie in obigen Beispiel). Auf die Grafik ist mittels \verb|\label| und \verb|\ref| im Text Bezug zu nehmen!

\begin{figure}[htbp]
  \begin{center}
    \includegraphics[width=.45\textwidth]{zih_logo_de_sw}
    \caption{Das ZIH-Logo}
    \label{fig:zhr_logo}
  \end{center}
\end{figure}


F\"ur die Tabellenumgebung (\verb|\begin{table} ... \end{table}|) gilt selbiges wie f\"ur die Grafiken, jedoch ist hier die Beschriftung oberhalb der Tabelle zu positionieren.

\subsection{Literaturverweise}
Literaturverweise innerhalb des Dokuments sollen ordentlich mit \verb|\cite| oder auch den extra zur Verf\"ugung gestellten Erweiterungen erfolgen. 
Die Literatur ist dabei in einer externen Bib-Datei zu halten. 
Dieses Dokument und die zugeh\"origen Quelldateien (\texttt{doku.tex} und \texttt{doku.bib}) k\"onnen dabei als Vorlage dienen.

\subsection{Dokumentaufteilung}
Es empfiehlt sich, das Dokument in kleinere Teile (vielleicht f\"ur jede Section) zu zerlegen und diese per \verb|\input| in ein Hauptdokument einzubinden. 
Dies vermeidet ein ellenlanges Hauptdokument und erleichtert auch die Fehlersuche.

\subsection{Symbol- / Abk\"urzungsverzeichnisse}
\label{sec:nomencl}
Mit dem Paket \texttt{nomencl} k\"onnen Symbolverzeichnisse erzeugt werden. 
Dies ist bereits in diese Vorlage mit integriert und direkt verwendbar. 
Dazu muss nur die Dokumentoption \texttt{nomencl} hinzugef\"ugt werden. 
Dies erzeugt das Symbolverzeichnis automatisch nach dem Inhaltsverzeichnis.
Mit dem Befehl \verb|\setnomenclmargin{<length>}| l\"asst sich die Breite der Spalte f\"ur die Abk\"urzungen anpassen. 
Mit \verb|\nomenclature{<Symbol>}{<Erkl\"arung>}| lassen sich Eintr\"age in das Verzeichnis aufnehmen. 
Das Verzeichnis muss manuell mit der makeindex-Umgebebung erstellt werden. 
Folgender Aufruf auf der Kommandozeile (im Verzeichnis mit der Hauptdatei erledigt dies:\\
\verb|makeindex <Hauptdatei>.nlo -s nomencl.ist -o <Hauptdatei>.nls|

\subsection{Umlaute}
\label{sec:umlaute}
Umlaute innerhalb von Textdokumenten stellen immer ein kleines Problem bei der Kompatibilit\"at dar. 
Dadurch, dass diese Vorlage \"uber \texttt{inputenc} die Eingabe von Umlauten direkt erm\"oglicht, l\"asst sich die Handhabung von Umlauten innerhalb der Dokumente vereinfachen.
\textbf{Folgendes ist aber zu beachten:} Abh\"angig vom Betriebssystem werden Umlaute unterschiedlich kodiert: Windows ISO-8559-1, Mac OS und Linux UTF-8.
Die Windows {\LaTeX}-Umgebungen k\"onnen aber heutzutage auch mit UTF-8 umgehen. 
Auf die richtige Dokumentoption ist zu achten!

\subsection{Kleinigkeiten}
Ein letztes Anliegen ist die bessere Verwendung von speziellen Leer- und Trennzeichen. 
Es sei an dieser Stelle auf folgenden Sonderzeichen in {\LaTeX} hingewiesen:
\begin{itemize}
	\item \verb|~| : Ein gesch\"utztes Leerzeichen. {\LaTeX} wird hier nicht zu einer neuen Zeile umbrechen.
	\item \verb|\-| : Ein Trennvorschlag. Zu verwenden bei falsch getrennten Worten. Einfach an den ``richtigen'' Stellen einf\"ugen.
	\item \verb|\,| : Trennzeichen zwischen Abk\"urzungen. Ist zum Beispiel zwischen z.\,B., u.\,\"a. oder O.\,B.\,d.\,A. einzuf\"ugen.
\end{itemize}

\section{Anregungen, Fehler und Verbesserungsvorschl\"age}
... werden nat\"urlich gern entgegengenommen -- am besten per Mail an:\\
\url{servicedesk@tu-dresden.de} mit ``ZIH-Latex-Vorlage'' in der Betreffzeile

%%%%%%%%%%%%%%%%%%%%%%%%%%%%%%%%%%%%%%%%%%%%%%%%%%%%%%%%%%%%%%%%%%%

\end{document}
